%%%%%%%%%%%%%%%%%%%%
% custom definitions
%%%%%%%%%%%%%%%%%%%%

% additional packages
\usepackage{setspace}
\usepackage{afterpage}
\usepackage{xcolor}

% increase spacing to 1 1/2
\onehalfspacing

% no space on start of paragraph
\setlength{\parindent}{0em}

% Tikz style definitions
\tikzstyle{decision} = [diamond, draw, fill=blue!20, text width=4.5em, text badly centered, node distance=3cm, inner sep=0pt]
\tikzstyle{block} = [rectangle, draw, fill=blue!20, text width=5em, text centered, rounded corners, minimum height=4em]
\tikzstyle{aufpunkt} = [rectangle, draw, fill=gray!20, text width=5em, text centered, rounded corners, minimum height=4em]
\tikzstyle{cloud} = [draw, ellipse,fill=red!20, node distance=3cm, minimum height=3em]

% solidity code highlighting in listings
\usepackage{listings, xcolor}

\definecolor{verylightgray}{rgb}{.97,.97,.97}

\lstdefinelanguage{Solidity}{
	keywords=[1]{anonymous, assembly, assert, balance, break, call, callcode, case, catch, class, constant, continue, constructor, contract, debugger, default, delegatecall, delete, do, else, emit, event, experimental, export, external, false, finally, for, function, gas, if, implements, import, in, indexed, instanceof, interface, internal, is, length, library, log0, log1, log2, log3, log4, memory, modifier, new, payable, pragma, private, protected, public, pure, push, require, return, returns, revert, selfdestruct, send, solidity, storage, struct, suicide, super, switch, then, this, throw, transfer, true, try, typeof, using, value, view, while, with, addmod, ecrecover, keccak256, mulmod, ripemd160, sha256, sha3}, % generic keywords including crypto operations
	keywordstyle=[1]\color{blue}\bfseries,
	keywords=[2]{address, bool, byte, bytes, bytes1, bytes2, bytes3, bytes4, bytes5, bytes6, bytes7, bytes8, bytes9, bytes10, bytes11, bytes12, bytes13, bytes14, bytes15, bytes16, bytes17, bytes18, bytes19, bytes20, bytes21, bytes22, bytes23, bytes24, bytes25, bytes26, bytes27, bytes28, bytes29, bytes30, bytes31, bytes32, enum, int, int8, int16, int24, int32, int40, int48, int56, int64, int72, int80, int88, int96, int104, int112, int120, int128, int136, int144, int152, int160, int168, int176, int184, int192, int200, int208, int216, int224, int232, int240, int248, int256, mapping, string, uint, uint8, uint16, uint24, uint32, uint40, uint48, uint56, uint64, uint72, uint80, uint88, uint96, uint104, uint112, uint120, uint128, uint136, uint144, uint152, uint160, uint168, uint176, uint184, uint192, uint200, uint208, uint216, uint224, uint232, uint240, uint248, uint256, var, void, ether, finney, szabo, wei, days, hours, minutes, seconds, weeks, years},	% types; money and time units
	keywordstyle=[2]\color{teal}\bfseries,
	keywords=[3]{block, blockhash, coinbase, difficulty, gaslimit, number, timestamp, msg, data, gas, sender, sig, value, now, tx, gasprice, origin},	% environment variables
	keywordstyle=[3]\color{violet}\bfseries,
	identifierstyle=\color{black},
	sensitive=false,
	comment=[l]{//},
	morecomment=[s]{/*}{*/},
	commentstyle=\color{gray}\ttfamily,
	stringstyle=\color{red}\ttfamily,
	morestring=[b]',
	morestring=[b]"
}

\lstset{
	language=Solidity,
	backgroundcolor=\color{verylightgray},
	extendedchars=true,
	basicstyle=\footnotesize\ttfamily,
	showstringspaces=false,
	showspaces=false,
	numbers=left,
	numberstyle=\footnotesize,
	numbersep=9pt,
	tabsize=2,
	breaklines=true,
	showtabs=false,
	captionpos=b
}


% New command definitions
% If you need help: https://www.overleaf.com/learn/latex/Commands
\newcommand{\vader}{\textsc{vader}\,}
\newcommand{\usdv}{\textsc{usdv}\,}
\newcommand{\aphra}{\textsc{aphra}\,}
\newcommand{\frax}{\textsc{frax}\,}
\newcommand{\fxs}{\textsc{fxs}\,}
\newcommand{\bean}{\textsc{bean}\,}
\newcommand{\eth}{\textsc{eth}\,}
\newcommand{\cvx}{\textsc{cvx}\,}
\newcommand{\crv}{\textsc{crv}\,}
\newcommand{\rpl}{\textsc{rpl}\,}
\newcommand{\yfi}{\textsc{yfi}\,}

% Custom colors
% From: http://latexcolor.com/
\definecolor{airforceblue}{rgb}{0.36, 0.54, 0.66}
\definecolor{amber}{rgb}{1.0, 0.75, 0.0}
\definecolor{amaranth}{rgb}{0.9, 0.17, 0.31}
\definecolor{darkcyan}{rgb}{0.0, 0.55, 0.55}
\definecolor{vegasgold}{rgb}{0.77, 0.7, 0.35}
\definecolor{vanilla}{rgb}{0.95, 0.9, 0.67}
\definecolor{unmellowyellow}{rgb}{1.0, 1.0, 0.4}
\definecolor{ufogreen}{rgb}{0.24, 0.82, 0.44}
\definecolor{ticklemepink}{rgb}{0.99, 0.54, 0.67}

% Tokenomics pie chart
\usetikzlibrary{fadings}
\tikzset{pics/wedge/.style={code={%
  \tikzset{wedge/.cd,#1}
  \def\kvw##1{\pgfkeysvalueof{/tikz/wedge/##1}}
  \pgfmathtruncatemacro{\itest}{3*(1+sign(sin(\kvw{alpha})))+1+sign(sin(\kvw{beta}))}
  \ifcase\itest
    %0: alpha>180,beta>180
       \draw[fill=\kvw{color},very thin]
                       (\kvw{alpha}:\kvw{radius}) 
                       -- ++(0,-\kvw{h}) arc(\kvw{alpha}:\kvw{beta}:\kvw{radius}) 
                       -- ++(0,\kvw{h})  arc(\kvw{beta}:\kvw{alpha}:\kvw{radius});
   \or
    %1: alpha>180,beta=0,180                       
       \draw[fill=\kvw{color},very thin]
                       (\kvw{alpha}:\kvw{radius}) 
                       -- ++(0,-\kvw{h}) arc(\kvw{alpha}:\kvw{beta}:\kvw{radius}) 
                       -- ++(0,\kvw{h})  arc(\kvw{beta}:\kvw{alpha}:\kvw{radius});
   \or
    %2: alpha>180,beta<180                     
       \draw[fill=\kvw{color},very thin]
                       (\kvw{alpha}:\kvw{radius}) 
                       -- ++(0,-\kvw{h}) arc(\kvw{alpha}:360:\kvw{radius}) 
                       -- ++(0,\kvw{h})  arc(360:\kvw{alpha}:\kvw{radius});
   \or
    %3: alpha=0,180,beta>180                       
       \draw[fill=\kvw{color},very thin]
                       (180:\kvw{radius}) 
                       -- ++(0,-\kvw{h}) arc(180:\kvw{beta}:\kvw{radius}) 
                       -- ++(0,\kvw{h})  arc(\kvw{beta}:180:\kvw{radius});
   \or
    %4: alpha=0,180,beta=0,180                     
       \draw[fill=\kvw{color},very thin]
                       (180:\kvw{radius}) 
                       -- ++(0,-\kvw{h}) arc(180:0:\kvw{radius}) 
                       -- ++(0,\kvw{h})  arc(0:180:\kvw{radius});
   \or
    %5: alpha=0,180,beta=<180                      
   \or
    %6: alpha<180,beta=>180                    
       \draw[fill=\kvw{color},very thin]
                       (180:\kvw{radius}) 
                       -- ++(0,-\kvw{h}) arc(180:\kvw{beta}:\kvw{radius}) 
                       -- ++(0,\kvw{h})  arc(\kvw{beta}:180:\kvw{radius});
   \or
    %7: alpha<180,beta=0,180
    \pgfmathtruncatemacro{\ibeta}{sign(cos(\kvw{beta}))}
    \ifnum\ibeta=1
        \draw[fill=\kvw{color},very thin] 
                        (180:\kvw{radius}) 
                        -- ++(0,-\kvw{h}) arc(180:360:\kvw{radius}) 
                        -- ++(0,\kvw{h})  arc(360:180:\kvw{radius});
    \fi
   \or
    %8: alpha<180,beta<180
    \pgfmathtruncatemacro{\ibeta}{sign(sin(\kvw{alpha})-sin(\kvw{beta}))}
    \ifnum\ibeta=1
       \draw[fill=\kvw{color},very thin]
                        (180:\kvw{radius}) 
                        -- ++(0,-\kvw{h}) arc(180:360:\kvw{radius}) 
                        -- ++(0,\kvw{h})  arc(360:180:\kvw{radius});
    \fi
  \fi
  \path[fill=\kvw{color},draw=black] (0,0)--
  (\kvw{alpha}:\kvw{radius})  arc(\kvw{alpha}:\kvw{beta}:\kvw{radius})
                                     --cycle;
}},
wedge/.cd,alpha/.initial=0,beta/.initial=0,%beta > alpha!
color/.initial=blue,
mix color/.initial=gray,radius/.initial=3cm,h/.initial=1cm,
/tikz/.cd,
pics/3d pie chart/.style={code={
  \def\kvw##1{\pgfkeysvalueof{/tikz/3d pie chart/##1}}
  \begin{scope}[yscale=\kvw{aspect},transform shape]
    \path[preaction={fill=black,opacity=.8,
           path fading=circle with fuzzy edge 20 percent}] 
           (0,-\kvw{h}-\kvw{radius}/4.5) 
           circle[radius=1.05*\kvw{radius}];
    \pgfmathsetmacro{\mysum}{0}      
    \foreach \XX/\ZZ  in {#1}  
    {\pgfmathsetmacro{\mysum}{\mysum+\XX}
     \xdef\mysum{\mysum}}
    \pgfmathsetmacro{\myangle}{\kvw{alpha0}}
    \foreach \XX/\ZZ [count=\YY starting from 0,remember=\myangle as \myangle] in {#1} 
    {\pgfmathsetmacro{\myangleB}{\myangle+\XX*(360/\mysum)}
     \pgfmathsetmacro{\mycolor}{{\kvw{colors}}[\YY]}
     \pic{wedge={alpha=\myangle,beta=\myangleB,color=\mycolor,
        radius/.expanded=\kvw{radius},
        h/.expanded=\kvw{h}
        }};
     \fill (\myangle/2+\myangleB/2:\kvw{radius}*\kvw{eccentricity})
      coordinate (\kvw{cname}-\YY) circle[radius=2pt];
     \pgfmathtruncatemacro{\mysign}{sign(cos(\myangle/2+\myangleB/2))} 
     \draw[thick] (\kvw{cname}-\YY)  -- 
      ++(\myangle/2+\myangleB/2:\kvw{armA}) -- ++ 
      (\mysign*3,0)
      \ifnum\mysign<0
        node[above right,transform shape=false]{\ZZ}
        node[below right,transform shape=false]{\XX\%}
      \else
        node[above left,transform shape=false]{\ZZ}
        node[below left,transform shape=false]{\XX\%}
      \fi;    
     \pgfmathsetmacro{\myangle}{\myangleB}
    }
    \shade[left color=black,middle color=white,right color=gray,opacity=0.4]
                          (180:\kvw{radius}) 
                          -- ++(0,-\kvw{h}) arc(180:360:\kvw{radius}) 
                          -- ++(0,\kvw{h})  arc(360:180:\kvw{radius});
  \end{scope}                       
}},
3d pie chart/.cd,
colors/.initial={"blue","red","orange","green","yellow"},
radius/.initial=3cm,h/.initial=1cm,alpha0/.initial=0,
aspect/.initial=0.6,eccentricity/.initial=0.7,cname/.initial=c,
armA/.initial=2cm,armB/.initial=3cm
}

% Timeline
\usepackage{tabularx}
\newcommand\ytl[3]{
\parbox[b]{8em}{\hfill{\color{gray}\bfseries\sffamily #1}~~~}\makebox[0pt][c]{}\vrule\quad \parbox[c]{4.5cm}{\vspace{7pt}\color{airforceblue}\raggedright\sffamily \phantom{0} \textbf{#2} \quad \quad \textbf{#3} \\[7pt]}\\[-3pt]}

